\documentclass[12pt,a4paper]{article}
\usepackage[utf8]{inputenc}
\usepackage[T2A]{fontenc}
\usepackage[russian]{babel}
\usepackage{amsmath}
\usepackage{amsfonts}
\usepackage{amssymb}
\usepackage{graphicx}
\usepackage{epstopdf}
%\usepackage[dvips]{graphics}
%\usepackage[dvips]{graphicx}
\usepackage{color}

%\newcommand{\bv}[1]{\mbox{\boldmath${#1}$}}
%\newcommand{\bkap}{\mbox{\boldmath$\kappa$}}
%\newcommand{\fc}{f_\mathrm{ce}}
%\newcommand{\fp}{f_\mathrm{pe}}
%\newcommand{\omp}{\omega_\mathrm{pe}}
%\newcommand{\df}{\Delta f}
%\newcommand{\fdm}{f_{0~\mathrm{\overline{DM}}}}
%\newcommand{\fum}{f_{0~\mathrm{\overline{UM}}}}
%\newcommand{\fbum}{f_{0~\mathrm{BUM}}}
%\newcommand{\dfbum}{\Delta f_{\mathrm{BUM}}}
%\newcommand{\fres}{f_\mathrm{res}}
%\newcommand{\dfdm}{\Delta f_\mathrm{DM}}
%\newcommand{\dfddm}{\Delta f_\mathrm{2DM}}
%\newcommand{\dfum}{\Delta f_\mathrm{UM}}

\bibliographystyle{unsrt}

\begin{document}

\begin{center}
	{\large \textbf{Пространственные характеристики искусственного свечения ионосферы в линии 630 нм при ВЧ воздействии радиоизлучением стенда "Сура"}}\\\ \\
	
	А. В. Шиндин, В.В. Клименко, С.М. Грач, Д.А. Когогин, И.А. Насыров, А.Б. Белецкий, Е.Н. Сергеев\\	

\end{center}

\textbf{Аннотация.}
 
В работе описана методика и представлены результаты моделирования распределения возбужденных атомов кислорода, которое обеспечивает генерацию искусственного свечения ионосферы в линии 630 нм и согласуется с данными, полученными в ходе экспериментов по ВЧ нагреву ионосферы мощной электромагнитной волной О-поляризации передатчиками стенда <<Сура>> 24, 26 августа 2014 года и 29 августа 2016 года. Результаты сопоставлены с данными вертикального зондирования ионосферы с помощью ионозондов. Выявлено, что ... \\

А.В. Шиндин. Нижегородский государственный университет им. Н.И. Лобачевского,
радиофизический факультет.\\
Пр. Гагарина 23, Нижний Новгород, Россия, 603950.\\
e-mail: shindin@rf.unn.ru; тел. (831) 465-61-27.\\

В.В. Клименко. Институт прикладной физики Российской Академии наук.\\
Ул. Ульянова, 46, Нижний Новгород, Россия, 603155.\\
email: klimenko@appl.sci-nnov.ru\\

С.М. Грач. Нижегородский государственный университет им. Н.И.
Лобачевского, радиофизический факультет.\\
Пр. Гагарина 23, Нижний Новгород, Россия, 603950.\\
e-mail: sgrach@rf.unn.ru; тел. (831) 465-61-27.\\

Д.А. Когогин. Казанский (Приволжский) федеральный университет, институт физики.\\
Ул. Кремлевская, 18, Казань, Респ. Татарстан, Россия, 420000.\\
email: dkogogin@kpfu.ru\\

И.А. Насыров. Казанский (Приволжский) федеральный университет, институт физики.\\
Ул. Кремлевская, 18, Казань, Респ. Татарстан, Россия, 420000.\\
email: igor.nasyrov@kpfu.ru\\

А.Б. Белецкий. Институт солнечно-земной физики Сибирского отделения Российской академии наук.\\
Ул. Лермонтова, 126-а, Иркутск, Россия, 664033.\\
email: beletsky@mail.iszf.irk.ru\\

Е.Н. Сергеев. Нижегородский государственный университет им. Н.И. Лобачевского,
Научно-исследовательский радиофизический институт.\\
Ул. Б. Печерская 25а, Нижний Новгород, Россия, 603950.\\
e-mail: esergeev@nirfi.unn.ru\\

\section{Введение}
\label{sec:intro}

Искусственное свечение ионосферы (искусственное полярное сияние) возникает в F-области ионосферы под действием мощного электромагнитного воздействия в результате следующей цепочки явлений: 1) взаимодействие волны накачки О-поляризации на частоте $f_0$ меньшей критической частоты F2-слоя ионосферы ($f_{0}F2$) с ионосферной плазмой приводит к генерации плазменных волн в области отражения волны накачки; 2) плазменные волны эффективно ускоряют свободные электроны; 3) электроны, обладающие необходимой энергией, возбуждают определенные энергетические уровни нейтрального ионосферного газа; 4) в процессе релаксации возбужденного атома до основного состояния происходит высвечивание фотона. Основная наблюдаемая линия свечения на нагревном стенде <<Сура>> (Нижегородская область, посёлок Васильсурск) --- красная ($\lambda = 630$ нм), связанная с излучением уровня O($^1$D) атомарного кислорода с энергией возбуждения $\mathcal{E}_{ex} = 1,96$ эВ и временем жизни $\widetilde{\tau} = 107$ с.

Регистрация и анализ характеристик искусственного свечения ионосферы с начала 70-х годов 20 века \cite{BIONDI1970} применяется для диагностики возмущенной ВЧ радиоизлучением области ионосферы наряду с вертикальным/наклонным зондированием ионосферы, регистрацией искусственного радиоизлучения ионосферы и др. В случае регистрации искусственного свечения в нескольких разнесенных в пространстве пунктах появляется возможность оценить пространственные характеристики светящейся области, а при использовании подходящей модели --- определить параметры трехмерного распределения возбужденных атомов кислорода. Первые подобные исследования в красной линии ($\lambda = 630$ нм) были проведены на нагревном стенде, расположенном вблизи г. Боулдер (США, шт. Колорадо), в 1971 г. \cite{HASLETT1974}. Путем триангуляции было установлено, что высота максимального свечения находилась в диапазоне $280\pm15$ км и слабо зависела от расположения области взаимодействия волны накачки с плазмой. В \cite{Gustavsson2001}, по данным экспериментов, проведенных на стенде EISCAT (Тромсе, Норвегия, трехточечная регистрация) в 1999 г. с помощью решения обратной задачи выявлено, что высота максимальной интенсивности свечения находилась в диапазоне $230\div240$ км с характерными вертикальным и горизонтальным масштабами 20 км. Форма (изоповерхность) распределения возбужденных атомов кислорода при этом варьировалась от сплющенной до вытянутой вдоль геомагнитного поля. Подобные измерения проводились также на нагревных стендах HIPAS \cite{Gustavsson2008} и HAARP \cite{Pedersen2011}. А в экспериментах, рассмотренных в \cite{Gustavsson2008}, максимум свечения в красной линии 630 нм располагался, согласно решению обратной задачи, на 20 км ниже высоты верхнегибридного резонанса волны накачки (где $f_0 = \sqrt{f_p^2 + f_{ce}^2}$, $f_p=\sqrt{\frac{e^2N}{\pi m}}$ - электронная плазменная частота, $f_{ce}=\frac{e H}{2\pi m c}$ - электронная циклотронная частота). В экспериментах, проведенных на стенде HAARP \cite{Pedersen2011} было зарегистрировано искусственное свечение в линии 557,7 нм от двух искусственных слоев дополнительной ионизации на высотах 160 и 175 км соответственно (высоты определялись путем триангуляции).

Первые наблюдения искусственного свечения ионосферы на стенде <<Сура>> в красной линии 630 нм были проведены в 1990 г. \cite{BERNHARDTSCALESGRACHEtAl1991}. В последующих сериях экспериментов обнаружены такие эффекты как дрейф пятна в скрещённых электромагнитных полях \cite{BernhardtWongHubaEtAl2000}, расщепление пятна в страты, вытянутые вдоль геомагнитного поля \cite{GrachKoschYashnovEtAl2007, Grach2012}, эффект магнитного зенита \cite{Grach2012, Shindin2014}, эффект подавления фонового свечения \cite{Grach2012, Shindin2014, Klimenko2017}. В 2014 и 2016 гг. на стенде <<Cура>> были проведены первые успешные эксперименты по регистрации искусственного свечения ионосферы в линии $\lambda = 630$ нм в двух разнесенных в пространстве пунктах наблюдения. Это первая двухточечная регистрация искусственного свечения на среднеширотном нагревном стенде.

В разделе \ref{sec:exp_setup} настоящей статьи описана постановка экспериментов и приведены характеристики регистрирующего оборудования. Далее детально описана методика предварительной обработки экспериментальных данных, включающая астрометрическую и спектрофотометрическую калибровки изображений (разделы \ref{sec:astro_cal} и \ref{sec:spectro_cal} соответственно). Необходимость использования нестандартных методик связана с особенностями экспериментов, значительными различиями в технических характеристиках используемых ПЗС-камер, а также необходимостью выделить на изображениях пятна искусственного свечения, интенсивность которых порой составляла <1\% от фонового свечения. Вопросы предобработки в подобных публикациях обычно либо излагаются поверхностно \cite{Gustavsson2008}, либо опускаются вовсе \cite{Pedersen2011}. Тем не менее результат предобработки во многом определяет успех последующего моделирования светящейся области. Раздел \ref{sec:modelling} посвящен моделированию пространственного распределения концентрации возбужденных атомов кислорода O($^1$D) и решению обратной задачи определения параметров модели. В целом применяемый авторами подход к выбору моделей несколько отличается от подхода, используемого в \cite{Gustavsson2001, Gustavsson2008}, что также определяется особенностями эксперимента. В разделе \ref{sec:inv_problem_results} приводятся результаты моделирования в сопоставлении с данными вертикального зондирования ионосферы с помощью ионозондов. Заключительный раздел посвящен обсуждению результатов.

% 56,15$^{\circ}$ с.ш., 46,10$^{\circ}$ в.д., магнитное наклонение --- 72,16 $^{\circ}$
\section{Постановка экспериментов} \label{sec:exp_setup}
Эксперименты были проведены 24 и 26 августа 2014 г. и 29 августа 2016 г. на стенде <<Сура>> (географические координаты 56,15$^{\circ}$ с.ш., 46,10$^{\circ}$ в.д., магнитное склонение 12,6$^{\circ}$,магнитное наклонение 72,1$^{\circ}$ --- по данным IGRF-12 \cite{Thebault2015} на уровне земной поверхности).Воздействие на ионосферу осуществлялось с помощью КВ радиоизлучения обыкновенной поляризации при вертикальной ориентации диаграммы направленности стенда (технические характеристики излучающей системы стенда <<Сура>> могут быть найдены например в \cite{BERNHARDTSCALESGRACHEtAl1991}). Использовались следующие частоты воздействия (частоты волны накачки, $f_0<f_{0}F2$): 24.08.14 --- 4740 кГц (эффективная излучаемая мощность $P_{eff} \approx 285$ МВт), 26.08.14 --- 5640 ($P_{eff} \approx 380$ МВт) и 4410 кГц ($P_{eff} \approx 260$ МВт), 29.08.16 --- 4300 ($P_{eff} \approx 225$ МВт) и 4350 кГц ($P_{eff} \approx 235$ МВт). Воздействие осуществлялось в импульсном режиме с периодом 6 минут и длительностью импульса 3 минуты. Во время пролета GPS спутников над возмущенной областью ионосферы период и длительность импульса изменялись на 8 и 12 минут соответственно.
% 24 августа 2014, мощности П1 - 130 кВт, П2 - 260 кВт, П3 - 110 кВт (4740 МГц)
% 26 августа 2014, мощности П1 - 150 кВт, П2 - 230 кВт, П3 - 170 кВт (5640 МГц); П1 - 130 кВт, П2 - 220 кВт, П3 - 130 кВт (4410 МГц)
% 29 августа 2016, мощности П1 - 130 кВт, П2 - 140 кВт, П3 - 150 кВт (4300 МГц); П1 - 130 кВт, П2 - 160 кВт, П3 - 140 кВт (4350 МГц)
% Belov, I.F., et al., The "SURA" experimental system for studying artificial disturbances in the ionosphere, Preprint No. 167, Scientific Research Radiophysics Institute (NIRFI), (in Russian; English translation available from P.A. Bernhardt), Gorky, 1983.
%  Белов И.Ф., Бычков В.В., Гетманцев Г.Г., Митяков Н.А., Пашкова Г.С. Экспериментальный комплекс «Сура» для исследования искусственных возмущений ионосферы // Препринт №167. Горький: НИРФИ, 1983.

\section{Астрометрическая калибровка изображений} \label{sec:astro_cal}

\section{Спектрофотометрическая калибровка изображений и выделение пятна искусственного свечения} \label{sec:spectro_cal}

\section{Моделирование функции распределения концентрации возбужденных атомов кислорода} \label{sec:modelling}

\section{Результаты решения обратной задачи} \label{sec:inv_problem_results}

\section{Обсуждение результатов} \label{sec:discuss}


\bibliography{citations}

\end{document}
