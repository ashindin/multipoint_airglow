  \documentclass[12pt,a4paper]{article}
%\documentclass[12pt]{article}
\usepackage[utf8]{inputenc}
\usepackage[russian]{babel}
\usepackage{amsmath}
\usepackage{amsfonts}
\usepackage{amssymb}
\usepackage{graphicx}
\graphicspath{{../figures/}}
\usepackage{epstopdf}
%\usepackage[dvips]{graphics}
%\usepackage[dvips]{graphicx}
\usepackage{color}
%\usepackage{hyperref}
\usepackage{url}

\usepackage{titlesec}
\bibliographystyle{unsrt}


\newcommand{\angstrom}{\textup{\AA}}

\setlength{\hoffset}{1.5 cm}
\setlength{\voffset}{-2 em}
\setlength{\topmargin}{0 cm}
\setlength{\headheight}{1 em}
\setlength{\headsep}{1 em}
\setlength{\oddsidemargin}{0 cm}
\setlength{\textwidth}{16 cm}
\setlength{\textheight}{24.7 cm}
\pagestyle{myheadings}
\renewcommand{\vec}{\mathbf}

\begin{document}
%\baselineskip = 24.00 pt

\begin{center}
	{\large\bf ПРОСТРАНСТВЕННЫЕ ХАРАКТЕРИСТИКИ ОБЛАСТИ ГЕНЕРАЦИИ ИСКУССТВЕННОГО СВЕЧЕНИЯ ИОНОСФЕРЫ В ЛИНИИ 630 НМ ПРИ ВОЗДЕЙСТВИИ РАДИОИЗЛУЧЕНИЕМ СТЕНДА <<СУРА>>}\\
	\vskip 0.5cm
	А. В. Шиндин\textsuperscript{1}, В.В. Клименко\textsuperscript{1,2}, Д.А. Когогин\textsuperscript{1,3}, А.Б. Белецкий\textsuperscript{1,4}, С.М. Грач\textsuperscript{1}, И.А. Насыров\textsuperscript{1,3}, Е.Н. Сергеев\textsuperscript{1}\\
	\vskip 0.5cm
	\textsuperscript{1}Нижегородский государственный университет им. Н.И.~Лобачевского,
	603950, г.~Нижний Новгород, пр.~Гагарина, 23\\
	\textsuperscript{2}Институт прикладной физики РАН,
	603950, г.~Нижний Новгород, ул.~Ульянова, 46\\
	\textsuperscript{3}Казанский (Приволжский) федеральный университет,
	420000, г.~Казань, ул.~Кремлевская, 18\\
	\textsuperscript{4}Казанский (Приволжский) федеральный университет,
	664033, г.~Иркутск, ул.~Лермонтова, 126-а\\
\end{center}
\thispagestyle{empty}
\textbf{Аннотация.}
 
В работе описана методика и представлены результаты моделирования распределения 
возбуждённых атомов кислорода, которое обеспечивает генерацию искусственного 
свечения ионосферы в линии 630 нм и согласуется с данными, полученными в ходе 
экспериментов по ВЧ нагреву ионосферы мощной электромагнитной волной 
О-поляризации передатчиками стенда <<Сура>>. Результаты сопоставлены с данными 
вертикального зондирования ионосферы с помощью ионозондов. Установлено, что в 
ходе экспериментов высота области свечения находилась на высоте $\sim250$ км 
и не зависела от высоты резонансов волны накачки. Характерный размер области 
составлял  $\sim35$ км, а форма хорошо описывалась наклонным сфероидом или 
каплевидным распределением. Среднее за время эксперимента значение максимальной 
концентрации возбуждённых атомов составило $\sim1200$ см$^{-3}$.\\

%А. В. Шиндин. Нижегородский государственный университет им. Н.И.~Лобачевского,
%радиофизический факультет.\\
%Пр. Гагарина 23, Нижний Новгород, Россия, 603950.\\
%e-mail: shindin@rf.unn.ru; тел. (831) 465-61-27.\\
%
%В.В. Клименко. Институт прикладной физики Российской Академии наук.\\
%Ул. Ульянова, 46, Нижний Новгород, Россия, 603155.\\
%email: klimenko@appl.sci-nnov.ru\\
%
%С.М. Грач. Нижегородский государственный университет им. Н.И.
%Лобачевского, радиофизический факультет.\\
%Пр. Гагарина 23, Нижний Новгород, Россия, 603950.\\
%e-mail: sgrach@rf.unn.ru; тел. (831) 465-61-27.\\
%
%Д.А. Когогин. Казанский (Приволжский) федеральный университет, институт физики.\\
%Ул. Кремлевская, 18, Казань, Респ. Татарстан, Россия, 420000.\\
%email: dkogogin@kpfu.ru\\
%
%И.А. Насыров. Казанский (Приволжский) федеральный университет, институт физики.\\
%Ул. Кремлевская, 18, Казань, Респ. Татарстан, Россия, 420000.\\
%email: igor.nasyrov@kpfu.ru\\
%
%А.Б. Белецкий. Институт солнечно-земной физики Сибирского отделения Российской академии наук.\\
%Ул. Лермонтова, 126-а, Иркутск, Россия, 664033.\\
%email: beletsky@mail.iszf.irk.ru\\
%
%Е.Н. Сергеев. Нижегородский государственный университет им. Н.И. Лобачевского,
%Научно-исследовательский радиофизический институт.\\
%Ул. Б. Печерская 25а, Нижний Новгород, Россия, 603950.\\
%e-mail: esergeev@nirfi.unn.ru\\

\section{Введение}
\label{sec:intro}

Искусственное свечение ионосферы наблюдается в F-области ионосферы под воздействием мощного электромагнитного излучения в результате следующей цепочки явлений: 1) взаимодействие волны накачки О-поляризации на частоте $f_0$ меньшей критической частоты F2-слоя ионосферы ($f_{0}F2$) с ионосферной плазмой приводит к генерации плазменных волн в области отражения волны накачки; 2) плазменные волны эффективно ускоряют свободные электроны; 3) электроны, обладающие необходимой энергией, возбуждают определённые энергетические уровни нейтрального ионосферного газа; 4) в процессе релаксации возбуждённого атома до основного состояния происходит высвечивание фотона. Основная наблюдаемая линия свечения на нагревном стенде <<Сура>> (Нижегородская область, посёлок Васильсурск) --- красная ($\lambda = 630$ нм), связанная с излучением уровня O($^1$D) атомарного кислорода с энергией возбуждения $\mathcal{E}_{ex} = 1,96$ эВ и временем жизни $\widetilde{\tau} = 107$ с.



%в результате возникновения области плазменной турбулентности, нагрева и ускорения электронов до э

С 1970-х годов регистрация и анализ характеристик искусственного свечения ионосферы применяется для диагностики возмущённой ВЧ радиоизлучением области ионосферы \cite{BIONDI1970} наряду с вертикальным/наклонным зондированием ионосферы, регистрацией искусственного радиоизлучения ионосферы и др. 
Большинство экспериментов по регистрации свечения проводится в одной точке, расположенной вблизи нагревного стенда. В этом случае определить пространственные параметры области свечения нельзя. Тем ни менее важен вопрос о том как соотносится высота области свечения с областями плазменных резонансов, так как это характеризует энергетику электронов.
В случае регистрации искусственного свечения в нескольких разнесённых в пространстве пунктах появляется возможность оценить пространственные характеристики светящейся области, а при использовании подходящей модели --- определить параметры трёхмерного распределения возбуждённых атомов кислорода. 

Согласно \cite{Mantas1996} омический нагрев плазмы и связанное с ним свечение в «красной» линии (630 нм) пространственно практически совпадают с областью плазменной турбулентности в окрестности высоты отражения мощной радиоволны. Если свечение связано с надтепловыми электронами ($\varepsilon \gtrsim 2$ эВ), ускоренными плазменными волнами, то как предсказывает теория \cite{Vaskov1983,Gurevich1985}, максимум яркости свечения должен быть смещён вниз (в сторону более плотной атмосферы) от области ускорения в турбулентном слое на расстояние порядка нескольких длин свободного пробега электронов с энергией $\varepsilon \sim 2$ эВ, что и наблюдалось экспериментально в \cite{HASLETT1974,Gustavsson2001,Gustavsson2008}. Таким образом, отслеживая высоту максимума свечения, можно идентифицировать причину свечения – нагрев или ускорение.

Известно, что высотный профиль яркости свечения в полярном сиянии при торможении пучка электронов в плотных слоях атмосферы зависит от вертикального профиля плотности нейтральной атмосферы и энергетического спектра электронов в пучке. Следовательно, измерения вертикального профиля яркости искусственного свечения представляют интерес как метод исследования энергетических характеристик надтепловых электронов и экспериментальной проверки теоретических моделей ускорения и транспортировки ускоренных электронов вверх и вниз от турбулентного слоя.
Опубликованных на данный момент экспериментальных данных \cite{HASLETT1974, Gustavsson2001, Gustavsson2008, Pedersen2011} недостаточно для построения физической модели ускорения и распространения электронов из области плазменной турбулентности.
%Первые подобные исследования в красной линии ($\lambda = 630$ нм) были проведены на нагревном стенде, расположенном вблизи г. Боулдер (США, шт. Колорадо), в 1971 г. \cite{HASLETT1974}. Путём триангуляции было установлено, что высота максимального свечения находилась в диапазоне $280\pm15$ км и слабо зависела от расположения области взаимодействия волны накачки с плазмой. В \cite{Gustavsson2001}, по данным экспериментов, проведённых на стенде EISCAT (Тромсё, Норвегия, трёхточечная регистрация) в 1999 г. с помощью сопоставления модели с экспериментом выявлено, что высота максимальной интенсивности свечения находилась в диапазоне $230\div240$ км с характерными вертикальным и горизонтальным масштабами 20 км. Форма области свечения при этом варьировалась от сплющенной до вытянутой вдоль геомагнитного поля. Подобные измерения проводились также на нагревных стендах HIPAS \cite{Gustavsson2008} и HAARP \cite{Pedersen2011}. В экспериментах, рассмотренных в \cite{Gustavsson2008}, максимум свечения в красной линии 630 нм располагался на 20 км ниже высоты верхнегибридного резонанса волны накачки (где $f_0 = \sqrt{f_p^2 + f_{ce}^2}$, $f_p=\sqrt{\frac{e^2N}{\pi m}}$ - электронная плазменная частота, $f_{ce}=\frac{e H}{2\pi m c}$ - электронная циклотронная частота).

% В экспериментах, проведённых на стенде HAARP \cite{Pedersen2011} было зарегистрировано искусственное свечение в линии 557,7 нм от двух искусственных слоёв дополнительной ионизации на высотах 160 и 175 км соответственно (высоты определялись путём триангуляции).

%В последующих сериях экспериментов обнаружены такие эффекты как дрейф пятна в скрещённых электромагнитных полях \cite{BernhardtWongHubaEtAl2000}, расщепление пятна в страты, вытянутые вдоль геомагнитного поля \cite{GrachKoschYashnovEtAl2007, Grach2012}, эффект магнитного зенита \cite{Grach2012, Shindin2014}, эффект подавления фонового свечения \cite{Grach2012, Shindin2014, Klimenko2017}. 
%Существуют теоретические работы \cite{Vaskov1983,Gurevich1985}, в которых показано, что длина релаксации ускоренных электронов из резонансной области увеличивается с высотой.

Однопозиционные наблюдения искусственного свечения ионосферы в красной линии 630 нм проводятся на стенде <<Сура>> с 1990 г. \cite{BERNHARDTSCALESGRACHEtAl1991, BernhardtWongHubaEtAl2000, GrachKoschYashnovEtAl2007, Grach2012, Shindin2014, Klimenko2017}. 
В 2014 и 2016 гг. на стенде <<Cура>> были проведены первые успешные эксперименты по регистрации искусственного свечения ионосферы в линии $\lambda = 630$ нм в двух разнесённых в пространстве пунктах наблюдения.
Целью настоящей работы является исследование пространственных характеристик области искусственного свечения ионосферы в линии 630 нм, в частности, высотного профиля яркости свечения, высоты максимума яркости свечения и её соотношения с высотой диссипации волны накачки (высотой области плазменной ВЧ турбулентности).
Ввиду особенностей используемого оптического оборудования в статье рассмотрены также вопросы предварительной обработки экспериментальных данных.

В разделе \ref{sec:exp_setup} настоящей статьи описана постановка экспериментов и приведены характеристики оборудования. Методики астрометрической, фотометрической калибровок изображений звёздного неба, а также методика выделения на изображениях пятен искусственного свечения рассмотрены в разделах \ref{sec:astro_cal}, \ref{sec:spectro_cal}, \ref{sec:glow} соответственно. Пятном свечения будем называть область повышенной яркости изображения, связанную с искусственным свечением ионосферы.
%Необходимость использования нестандартных методик связана с особенностями экспериментов, значительными различиями в технических характеристиках используемых ПЗС-камер, а также малой интенсивностью (до <1\% от фоновых значений) пятен искусственного свечения. Вопросы предварительной обработки в подобных публикациях обычно либо излагаются поверхностно \cite{Gustavsson2008}, либо опускаются вовсе \cite{Pedersen2011}. Тем не менее ее результат во многом определяет успех последующего моделирования светящейся области. 
Раздел \ref{sec:modelling} посвящён моделированию пространственного распределения концентрации возбуждённых атомов кислорода O($^1$D) и решению обратной задачи определения параметров модели. В целом применяемый авторами подход к выбору моделей несколько отличается от подхода, используемого в \cite{Gustavsson2001, Gustavsson2008}, что также определяется особенностями эксперимента. В разделе \ref{sec:inv_problem_results} приводятся результаты моделирования в сопоставлении с данными вертикального зондирования ионосферы с помощью ионозонда. Заключительный раздел посвящён обсуждению результатов.

% 56,15$^{\circ}$ с.ш., 46,10$^{\circ}$ в.д., магнитное наклонение --- 72,16 $^{\circ}$
\section{Аппаратура и организация экспериментов} \label{sec:exp_setup}
Эксперименты были проведены 24 и 26 августа 2014 г. и 29 августа 2016 г. на стенде <<Сура>> (географические координаты 56,15$^{\circ}$ с.ш., 46,10$^{\circ}$ в.д., магнитное склонение 11,2$^{\circ}$, магнитное наклонение 71,9$^{\circ}$ --- по данным IGRF-12 \cite{Thebault2015} на уровне 250 км от земной поверхности). Воздействие на ионосферу осуществлялось с помощью КВ радиоизлучения обыкновенной поляризации при вертикальной ориентации диаграммы направленности стенда (технические характеристики излучающей системы стенда <<Сура>> могут быть найдены например в \cite{BERNHARDTSCALESGRACHEtAl1991}). Использовались следующие частоты воздействия (частоты волны накачки, $f_0<f_{0}F2$): 24.08.14 --- 4740 кГц (эффективная излучаемая мощность $P_{\text{эф}} \approx 85$ МВт), 26.08.14 --- 5640 ($P_{\text{эф}} \approx 125$ МВт) и 4410 кГц ($P_{\text{эф}} \approx 80$ МВт), 29.08.16 --- 4300 ($P_{\text{эф}} \approx 65$ МВт) и 4350 кГц ($P_{\text{эф}} \approx 70$ МВт). Воздействие осуществлялось в импульсном режиме с периодом 6 минут и длительностью импульса 3 минуты, а также с периодом 12 и длительностью 8 минут. Эксперименты сопровождались работой станций вертикального зондирования ионосферы CADI \cite{CADI_specs} и <<Циклон>> (КФУ).
% 24 августа 2014, мощности П1 - 130 кВт, П2 - 260 кВт, П3 - 110 кВт (4740 МГц)
% 26 августа 2014, мощности П1 - 150 кВт, П2 - 230 кВт, П3 - 170 кВт (5640 МГц); П1 - 130 кВт, П2 - 220 кВт, П3 - 130 кВт (4410 МГц)
% 29 августа 2016, мощности П1 - 130 кВт, П2 - 140 кВт, П3 - 150 кВт (4300 МГц); П1 - 130 кВт, П2 - 160 кВт, П3 - 140 кВт (4350 МГц)
% Belov, I.F., et al., The <<SURA>> experimental system for studying artificial disturbances in the ionosphere, Preprint No. 167, Scientific Research Radiophysics Institute (NIRFI), (in Russian; English translation available from P.A. Bernhardt), Gorky, 1983.
%  Белов И.Ф., Бычков В.В., Гетманцев Г.Г., Митяков Н.А., Пашкова Г.С. Экспериментальный комплекс «Сура» для исследования искусственных возмущений ионосферы // Препринт №167. Горький: НИРФИ, 1983.

Регистрация искусственного свечения производилась в двух пунктах: A), расположенный в непосредственной близости (на расстоянии $\sim850$м) от антенной системы стенда <<Сура>>; Б), расположенный на территории магнитной обсерватории КФУ (географические координаты 55,56$^{\circ}$ с.ш., 48,45$^{\circ}$ в.д., на расстоянии $\sim170$ км от нагревного стенда). В 2014 г. на пункте А измерения проводились с помощью цифровой светочувствительной ФПЗС-камерой S1C/079-FU (далее S1C) \cite{S1C_specs} (размер кадра $578\times578$ пикселей) со светосильным объективом Юпитер-НС NC2 \cite{NC2_specs} c полем зрения $\sim20^{\circ}$, фокусным расстоянием 35 мм и относительным отверстием 1:05. В 2016 г. на пункте A измерения проводились с помощью камеры SBIG-8300M (далее SBIG) \cite{SBIG_specs} (размер кадра $3326\times2504$ пикселей) с объективом Canon EF 50mm f/1.2L USM с полем зрения $20\times15^{\circ}$. На обеих камерах были установлены светофильтры на длину волны $\lambda=6300\angstrom$ с полосой пропускания $\Delta \lambda = 100\angstrom$. Пункт Б оснащался камерой KEO Sentinel (далее KEO) \cite{KEO_specs} (размер кадра $2048\times2048$ пикселей) c объективом с полем зрения $\sim150^{\circ}$ со светофильтром на длину волны $\lambda=6300\angstrom$ с полосой пропускания $\Delta \lambda = 20\angstrom$. 
Для увеличения чувствительности съёмка велась с использованием бинирования (объединение пикселей) $2\times2$, $9\times9$ и $4\times4$ для камер S1C, SBIG и KEO соответственно. В экспериментах 2014 г. длительность экспозиции камеры S1C составляла 15 с, камеры KEO --- 30 c. В экспериментах 2016 г. длительность экспозиции камер SBIG и KEO составляла 25 с.


% http://www.keoscientific.com/Documents/KeoSentinelBrochure.pdf
% http://www.sil.sk.ca/content/cadi
Относительное расположение стенда <<Сура>> и пунктов А и Б представлено на рисунке \ref{fig:fig1}.



\begin{figure}[h]
	\center{\includegraphics[width=1\linewidth]{fig1}}
	\caption{Карта расположения антенной системы стенда <<Сура>> и приёмных пунктов. По горизонтали и вертикали отложены географические долгота и широта в градусах соответственно.}
	\label{fig:fig1}
\end{figure}

\section{Астрометрическая калибровка изображений звёздного неба} \label{sec:astro_cal}
Астрометрическая калибровка нужна для того, чтобы связать координаты в пикселях изображения с угловыми координатами на небесной сфере. В астрономических исследованиях обычно используется либо Международная небесная система координат (International Celestial Reference System, ICRS), либо экваториальная система координат эпохи J2000, и на текущий момент существует как минимум два бесплатных программных пакета IzmCCD \cite{Izmccd,Izmailov2010} и Astrometry.net \cite{Astrometry.net, Lang2010} (последний является ещё и пакетом с открытым исходным кодом) реализующих астрономическую калибровку автоматически, анализируя взаимное расположение звёзд на снимке. Применяемая в этих пакетах методика зависит от типа проекции небесной сферы, используемого для получения плоского изображения \cite{Calabretta2002}. В оптической астрономии широко используется гномоническая проекция, и в этом случае оба пакета успешно справляются с астрометрической калибровкой. В камерах всего неба, оснащённых объективом типа <<рыбий глаз>>, как правило используется зенитная эквидистантная проекция, и при этом, ввиду присутствия геометрических искажений на периферии кадра, автоматическая астрометрическая калибровка сильно затруднена.

Так как область свечения вращается вместе с Землёй, а ориентация камер не менялась в течение эксперимента, то для целей настоящего исследования интерес представляет астрометрическая калибровка, связанная с определением по координатам пикселя изображения горизонтальных координат: высоты (угла места) и азимута объекта. И хотя нет никаких сложностей в переходе между системами небесных координат, подобного функционала на данный момент нет ни в одной известной авторам программе. Поэтому для астрометрической калибровки изображений в данной работе используется методика, изложенная в \cite{Montenbruck2000}, модифицированная нами для получения горизонтальных координат.
%Для камеры KEO, к сожалению, приходилось производить калибровку вручную.

С помощью пакета Astrometry.net автоматически составлялась таблица ярких (опорных) звёзд, присутствующих на снимке. Каждой звезде ставились в соответствие две пары координат: ($x,y$) в системе координат, связанной с плоскостью кадра и ($\alpha, \delta$) в экваториальной системе координат эпохи J2000. В случае камеры KEO для получения подобной таблицы требовалось вручную идентифицировать звезды на снимке и получить их экваториальные координаты из звёздного каталога. Затем экваториальные координаты преобразовывались в горизонтальные ($h,A$) высоту и азимут. Из полученной таблицы пар координат ($x,y$) -- ($h,A$) ярких звёзд можно получить функциональную связь между системами координат (см. Приложение \ref{app:astrometric}).

Средняя ошибка определения координат в пикселях составила 0,15 и 0,67 пикселя для камеры S1C в экспериментах 24.08.14 и 26.08.14 соответственно; 0,33, 0,39 и 0,53 пикселя для камеры KEO в экспериментах 24.08.14, 26.08.14 и 29.08.16; 0,1 пикселя для камеры SBIG в эксперименте 29.08.16. Корректность астрометрической калибровки легко проверить с помощью наложения на снимки ярких звёзд или линий созвездий (см. рисунок \ref{fig:fig2}).


\begin{figure}[h]
	\center{\includegraphics[width=1\linewidth]{fig2}}
	\caption{Портреты ночного неба, полученные с помощью камер SBIG (панель слева) и KEO (панель справа) 29.08.16 в 20:23:30 UTC. На каждом снимке зелёным цветом показана сетка горизонтальных небесных координат, красным --- направление на юг, жёлтым --- линии созвездий.}
	\label{fig:fig2}
\end{figure}


\section{Фотометрическая калибровка изображений звёздного неба} \label{sec:spectro_cal}
Фотометрическая калибровка заключается в расчёте коэффициента пропорциональности между двумя шкалами интенсивностей: шкалой относительных единиц АЦП светочувствительного элемента камеры и абсолютной шкалой поверхностной яркости. При бистатических наблюдениях такая калибровка даёт возможность определить физические параметры (концентрацию возбуждённых атомов кислорода) излучающей области. Ниже описывается методика спектрофотометрической калибровки ПЗС-камер по снимках звёздного неба с помощью спектрофотометрического каталога \cite{Kharitonov1978}.

В каталоге \cite{Kharitonov1978} приведены внеатмосферные квазимонохроматические освещённости $E(\lambda) [\frac{\text{эрг}}{\text{см}^2 \cdot \text{с} \cdot \text{см}}]$ для длин волн $\lambda$ в диапазоне $[3225,7975]\angstrom$, создаваемые спектрами 576 ярких звёзд. Плотность потока фотонов от звезды на границе атмосферы:
\begin{equation}\label{eq:eq9}
F(\lambda)=\frac{\lambda}{h \cdot c} E(\lambda),
\end{equation}
где $h$ --- постоянная Планка, $c$ -- скорость света. Поверхностная яркость изображения звезды:
\begin{equation}\label{eq:eq10}
B(\lambda) [\text{Рэлей}] = \frac {4\pi}{10^6} \cdot \frac{F(\lambda) \Delta \lambda} {\Omega_\text{s}},
\end{equation}
где $\Delta \lambda$ --- полоса пропускания светофильтра, а $\Omega_\text{s}$ --- эквивалентный телесный угол, соответствующий изображению звезды в фокальной плоскости. Пусть звезда занимает в кадре $n$ пикселей (для ярких звёзд из каталога \cite{Kharitonov1978} как правило $n>3$), тогда $\Omega_\text{s}=n\Omega_{\text{pix}},$ где $\Omega_{\text{pix}}$ --- телесный угол одного пикселя. Для широкоугольных объективов вообще говоря $\Omega_{\text{pix}}=\Omega_{\text{pix}}(\theta)$, где $\theta$ --- угол, отсчитываемый от направления оптической оси камеры в сторону периферии кадра. 

Для расчёта зависимости $\Omega_{\text{pix}}(\theta)$ введём угол $\varphi$, отсчитываемый от направления, параллельного вертикальной границе кадра. Заметим, что если камера сориентирована строго в зенит, а направление север-юг при этом параллельно боковой грани кадра, то угол $\theta=\pi / 2 - h = z$ совпадает с зенитным углом, а $\varphi = A$ --- с азимутом. Введём функцию $r(\theta)$ --- расстояние в пикселях от центра кадра до горизонтальной линии с зенитным углом $\theta$. Тогда $\rho(\theta)=dr/d\theta$ --- угловое разрешение камеры по зенитному углу (количество пикселей в единице зенитного угла). Конкретный вид функции $r(\theta)$ зависит от типа проекции небесной сферы, используемой в камере. Для гномонической проекции $r(\theta)=C \tan (\theta)$, а для зенитной эквидистантной $r(\theta)=C\theta$, где константа $C$ характеризует масштаб изображения и может быть определена в результате астрометрической калибровки ($C=\sqrt{c_2^2+d_2^2}$ или $C=\sqrt{c_3^2+d_3^2}$, см. Приложение \ref{app:astrometric}). Для камеры S1C в режиме бинирования $2\times2$ $C=13,69 \frac{\text{пиксель}}{^{\circ}}$, для камеры KEO в режиме $4\times4$ $C= 3,57 \frac{\text{пиксель}}{^{\circ}}$, для SBIG в режиме $9\times9$ --- $C=18,51 \frac{\text{пиксель}}{^{\circ}}$   Напишем далее выражение для телесного угла, образуемого конусом с углом раскрыва $2\theta$:
\begin{equation}\label{eq:eq11}
\Omega (\theta)= \int_{0}^{2\pi} \int_{0}^{\theta} \sin\vartheta d\vartheta d\varphi = \int_{0}^{\theta} 2\pi\sin{\vartheta} d\vartheta=2\pi(1-\cos\theta).
\end{equation}
Видно, что приращение телесного угла $d\Omega = 2\pi\sin{\theta}d\theta$. Получаем, что:  
\begin{equation}\label{eq:eq12}
\Omega_\text{pix} (\theta)= 2\pi\sin{\theta} \frac{\Delta\theta_\text{pix}(\theta)}{N(\theta)},
\end{equation}
где $\Delta\theta_\text{pix}(\theta)=1/\rho(\theta)$ --- приращение зенитного угла, покрывающее один пиксель, $N(\theta) = 2\pi r(\theta)$ --- число пикселей внутри кольца шириной один пиксель. В итоге для величины телесного угла, приходящегося на один пиксель изображения, следующее выражение:
\begin{equation}\label{eq:eq13}
\Omega_\text{pix} (\theta)= \frac{\sin{\theta}}{r(\theta) \frac{dr(\theta)}{d\theta}}.
\end{equation}
Подставляя (\ref{eq:eq9}) и (\ref{eq:eq13}) в (\ref{eq:eq10}), получим:
\begin{equation}\label{eq:eq14}
B(\lambda,\theta) = \frac {4\pi}{10^6} \frac{\lambda\Delta\lambda}{hc}\cdot \frac{E(\lambda) r(\theta)\frac{dr}{d\theta}} {n \sin\theta}.
\end{equation}

Для того, чтобы вычислить коэффициент пропорциональности $R$ между единицами АЦП и шкалой поверхностной яркости требуется: 1) после проведения стандартной предварительной обработки изображения (вычитания тёмнового кадра и учёта плоского поля) найти на снимке звезду из каталога \cite{Kharitonov1978}; 2) вычислить среднюю интенсивность $I$ всех $n$ пикселей изображения, связанных с этой звездой; 3) вычесть из полученного значение добавку, несвязанную со светом от звезды; 4) рассчитать поверхностную яркость звезды по формуле (\ref{eq:eq14}), взяв $\lambda = 6300\angstrom$. В итоге:
\begin{equation}\label{eq:eq15}
R [\frac{\text{Рэлей}}{\text{ед. АЦП}}]=\frac{B(\lambda,\theta)}{I}.
\end{equation}

Изложенная выше методика даёт возможность рассчитать коэффициент $R$ по одной звезде. Чаще всего на снимках присутствует сразу несколько звёзд из каталога \cite{Kharitonov1978}. Кроме того, в ходе эксперимента регистрируется большое количество изображений звёздного неба. Таким образом можно усреднить полученные значения коэффициента $R$ по достаточно большой выборке и оценить ошибку его определения. Пример такой оценки, проведённой в автоматическом режиме, для камеры S1C в эксперименте 26.08.14 приведён на рисунке \ref{fig:fig3}. Необходимо отметить, что коэффициент $R$ зависит от размера изображения (режима бинирования) и длительности выдержки, во всем остальном он характеризует камеру вместе с оптической системой, поэтому вариации коэффициента $R$ могут быть использованы для оценки прозрачности атмосферы. Присутствие даже слабой облачности приводит к увеличению коэффициента $R$. Например, в эксперименте, проведённом 26.08.14 в пункте А ясное небо было только в течение последней трети общей длительности эксперимента (см рисунок \ref{fig:fig3}).

\begin{figure}[h]
	\center{\includegraphics[width=1\linewidth]{fig3}}
	\caption{Динамика изменения калибровочного коэффициента $R [\frac{\text{пиксель}}{^{\circ}}]$ для камеры S1C в ходе эксперимента, проведённого 26.08.14 г. Красным цветом (медианное значение + СКО) показы значения для каждого снимка, а чёрным --- медианное значение, а также СКО пунктиром, по снимкам, зарегистрированным в интервале времени [20:08, 21:14].}
	\label{fig:fig3}
\end{figure}

В результате спектрофотометрической калибровки были получены следующие калибровочные коэффициенты: для камеры S1C --- $R = 2,20 \pm 0,15 \frac{\text{пиксель}}{^{\circ}}$, для камеры KEO в экспериментах 2014 г. (длительность экспозиции - 30 с) --- $R = 0,49 \pm 0,13 \frac{\text{пиксель}}{^{\circ}}$, для камеры KEO в эксперименте 2016 г. (длительность экспозиции - 25 с) --- $R = 0,60 \pm 0,16 \frac{\text{пиксель}}{^{\circ}}$, для камеры SBIG --- $R = 0,045 \pm 0,001\frac{\text{пиксель}}{^{\circ}}$.

\section{Выделение пятен искусственного свечения} \label{sec:glow}
Методика обработки изображений, которая применялась авторами в \cite{Grach2012}, состоит в выборе из всех зарегистрированных изображений т.н. опорных кадров, зарегистрированных перед очередным включением передатчиков. 
%Для всех остальных изображений, путём линейной интерполяции опорных кадров формировалось вычитаемое изображение. 
Так как в интервале между опорными кадрами интенсивность фонового свечения неба менялась, то для ее определения во время работы передатчиков применялась процедура линейной интерполяции по времени. Для компенсации движения отдельных звёзд от кадра к кадру вычитаемое фоновое изображение дополнительно поворачивалось. 
Вместе с медианной фильтрацией, применяемой для увеличения соотношения сигнал/шум, подобная методика позволяла получать окончательные изображения пятен искусственного свечения. Для камеры KEO, обладающей широким полем зрения, поворот изображения не даёт желаемых результатов из-за значительной дисторсии объектива. В данном исследовании вместо поворота вокруг полюса мира, использовалась цепочка преобразований координат, учитывающее тип проекции камеры (см. Приложение \ref{app:postprocessing}). 

Пример исходного изображения (камера KEO), а также изображения, полученного с помощью данной методики приведены на рисунке \ref{fig:fig5}.
\begin{figure}[h]
	\center{\includegraphics[width=1\linewidth]{fig5}}
	\caption{Изображение одного из наиболее ярких пятен искусственного свечения, зарегистрированное 26.08.2014 в 19:36:30 UTC	с помощью камеры KEO. Цветовая шкала --- поверхностная яркость в Рэлеях. Левая панель --- исходный снимок до обработки. Красным квадратом выделена область кадра, показанная на правой панели. Правая панель --- снимок после обработки, включающей удаление фонового свечения по опорным кадрам и медианной фильтрации по области $15 \times 15$ пикселей.}
	\label{fig:fig5}
\end{figure}
Отметим, что в экспериментах 2014 г. использовались разные длительности экспозиции камер (15 с --- для S1C, 30 с --- для KEO) и в целом регистрация кадров в камерах S1C и KEO не была синхронизирована. Поэтому кадры, полученные с помощью камеры S1C, объединялись по два. Разница между временами старта экспозиции между такими <<объединёнными>> кадрами с камеры S1C и кадрами с камеры KEO не превышала 7 с. После проведения обработки по вышеописанной методике, изображения, зарегистрированные в пунктах А и Б, группировались парами по времени старта экспозиции.

%\begin{figure}[h]
%	\center{\includegraphics[width=1\linewidth]{fig4}}
%	\caption{Панель а. Динамика медианного значения яркости по области в виде квадрата со стороной 31 пиксель зарегистрированная с помощью камеры S1C в эксперименте, проведенном 24.08.14 г. (красные точки). Значения, соответствующие опорным <<преднагревным>> кадрам выделены синим цветом. Линейная интерполяция яркостных значений показана синей линией. Панель б. Вариации поверхностной яркости, полученные путем вычитания интерполяционных фоновых значений из медианных значений яркости.}
%	\label{fig:fig4}
%\end{figure}

\section{Моделирование функции распределения концентрации возбуждённых атомов кислорода} \label{sec:modelling}
 
 В результате процедур, описанных в разделах \ref{sec:astro_cal} - \ref{sec:glow}, мы определённого момента времени получили два распределения поверхностной яркости $B_\text{А} (A,h)$ и $B_\text{Б} (A,h)$ по угловым горизонтальным координатам, зарегистрированных в пунктах А и Б соответственно в определённый момент времени. Поверхностная яркость связана с концентрацией $n$ возбуждённых атомов кислорода O($^1$D) следующим соотношением:
 \begin{equation}\label{eq:mod1}
 B [\text{Рэлей}] = \frac{A_{6300}}{10^6}\int_0^\infty n(l)dl,
 \end{equation}
где 
%$A_{6300}=5,627\cdot10^{-3}$
$A_{6300}=5,15\cdot10^{-3}$ % надо пересчитать концентрации с новым коэффициентом.
 \cite{Gustavsson2008} --- коэффициент Эйнштейна для длины волны $\lambda = 6300\angstrom$, а интегрирование ведётся вдоль луча зрения. Если дополнить горизонтальную систему координат длиной радиус-вектора $r$, и записать концентрацию $n$ в получившейся сферической системе координат с центром в точке наблюдения, то получим:
\begin{equation}\label{eq:mod2}
%\begin{cases}
 B_\text{А} (A,h) = \frac{A_{6300}}{10^6}\int_0^\infty n_\text{А}(A,h,r)dr,
 B_\text{Б} (A,h) = \frac{A_{6300}}{10^6}\int_0^\infty n_\text{Б}(A,h,r)dr
% \end{cases}
\end{equation}
Функцию концентрации $n$ возбуждённых атомов кислорода удобнее всего задавать в декартовой системе координат $\{x,y,z\}$ с началом в какой-либо выделенной точке распределения концентрации (обычно в точке с максимальной концентрацией), плоскостью $\{x,y\}$, параллельной земной поверхности, осями $x$ и $y$, направленными по сторонам света, и осью $z$, направленной в зенит. Таким образом определяется система координат ENU (East, North, Up), у которой ось $x$ направлена на восток, а ось $y$ --- на север. Требуется ввести ещё три параметра, чтобы полностью определить положение системы ENU относительно земной поверхности: широту $\varphi$, долготу $\lambda$ (не путать с длиной волны) и высоту $h$ над земной поверхностью (использовался эллипсоид WGS-84) начала отсчёта. Зададим модельное распределение концентрации в системе ENU (далее будем обозначать связь системы с моделью с помощью буквы M: ENU$_\text{M}$, а также соответствующие географические координаты начала отсчёта системы ENU$_\text{M}$ как $\varphi_\text{M}$, $\lambda_\text{M}$ и $h_\text{M}$) в виде:
\begin{equation}\label{eq:mod3}
n=n(x,y,z).
\end{equation}
%где $\vec{p}$ --- вектор параметров модели.
% Например, для модели 3-х мерного нормального распределения концентрации:
% \begin{equation}\label{eq:mod4}
% f(x,y,z,\vec{p})=p_1\exp{(-\frac{x^2 + y^2 + z^2}{p_2^2})},
% \end{equation}
%где параметр $p_1$ имеет смысл максимальной концентрации, а $p_2$ --- расстояния, на котором концентрация спадает в $e$ раз. 
Чтобы воспользоваться формулами (\ref{eq:mod2}), требуется выполнить переход из системы координат ENU$_\text{M}$, связанной с моделью, в системы ENU$_\text{А}$ и ENU$_\text{Б}$, с началом координат в пунктах А и Б соответственно. Такой переход удобно представить в виде двух последовательных переходов:
 \begin{equation}\label{eq:mod5}
 \text{ENU}_\text{M} \xrightarrow[]{M_1} \text{ECEF} \xrightarrow[]{M_2^\text{А}} \text{ENU}_\text{А}, \text{ENU}_\text{M} \xrightarrow[]{M_1} \text{ECEF} \xrightarrow[]{M_2^\text{Б}} \text{ENU}_\text{Б}, 
 \end{equation}
где ECEF (earth-centered, earth-fixed) --- декартова географическая система координат (начало координат --- центр масс Земли, ось $x$ направлена в сторону нулевого меридиана, а ось $z$ --- на опорный полюс), а $M_1$, $M_2^\text{А}$ и $M_2^\text{Б}$ соответствующие матрицы поворота.
Выражения для матриц поворота $M_1$, $M_2^\text{А}$ и $M_2^\text{Б}$, а также система для перехода от декартовой системы координат к горизонтальной приведены в Приложении \ref{app:modelling1}.
%Приведём далее выражения для матриц поворота:

Рассмотрим далее конкретные функции концентрации $n(x,y,z)$ возбуждённых атомов кислорода в системе координат ENU$_\text{M}$, используемые для моделирования светящейся области.

\subsection{Трёхмерное гауссово распределение} \label{subsec:model1}

Наиболее простая модель --- трёхмерное гауссово распределение (далее модель M1):
\begin{equation}\label{eq:gauss1}
n(x,y,z,\vec{p})=p_1 \exp{(-\frac{x^2+y^2+z^2}{p_2^2})},
\end{equation}
где $\vec{p}$ --- вектор параметров модели. Данная модель имеет 2 параметра $\vec{p}=\{p_1,p_2\}$: $p_1$ имеет смысл максимальной концентрации, а $p_2$ --- расстояния, на котором концентрация спадает в $e$ раз. Изоповерхности данного распределения представляют собой концентрические сферы (рассматриваемые в этой секции модели удобно классифицировать именно по форме изоповерхностей). 

Модель М1 является базовой для моделей М2, М3 и М4. Модель М2 (изоповерхности --- сфероиды):
\begin{equation}\label{eq:gauss2}
	n(x,y,z,\vec{p})=p_1 \exp{(-\frac{x^2+y^2}{p_2^2}-\frac{z^2}{p_3^2})},
\end{equation}
где параметр $p_3$ характеризует вытянутость изоповерхностей вдоль вертикальной оси.

Модель М3 (изоповерхности --- эллипсоиды):
\begin{equation}\label{eq:gauss3}
n(x,y,z,\vec{p})=p_1 \exp{(-\frac{x^2}{p_2^2}-\frac{y^2}{p_3^2}-\frac{z^2}{p_4^2})},
\end{equation}
где параметр $p_{1}$ имеет тот же смысл, что и у модели М2, а $p_{2-4}$ --- 
характеризуют скорость убывания концентрации вдоль соответствующих осей координат.

Модель М4 (изоповерхности --- сфероиды с наклонной осью):
\begin{equation}\label{eq:gauss4}
\begin{split}
n(x,y,z,\vec{p})=p_1 \text{ exp} \bigg[ 
-\frac{(x\cos{p_4}\cos{p_5} + y\sin{p_4}\cos{p_5} + z\sin{p_5})^2}{p_2^2}-\\
-\frac{(-x\sin{p_4} + y\cos{p_4})^2}{p_2^2} -\\ 
-\frac{(x\sin{p_5}\cos{p_4} + y\sin{p_4}\sin{p_5} - z\cos{p_5})^2}{p_3^2}\bigg],
\end{split}
\end{equation}
где параметры $p_{1-3}$ --- те же, что у модели М2, а $p_4$ и $p_5$ --- углы, характеризующие направление вертикальной оси сфероидов.

Из (\ref{eq:gauss1}-\ref{eq:gauss4}) видно, что модель M1 --- частный случай моделей М2--М4, а модель М2 --- частный случай моделей М3 и М4, модели М3 и М4 являются более общими. При этом вектор параметров $\vec{p}$ для модели М3 состоит из 4 компонент, а для модели М4 --- из 5.

Использование моделей М1-М4 допускает аналитическое вычисление несобственного интеграла в (\ref{eq:mod1}), которое может быть выполнено с помощью систем компьютерной алгебры (например, Maxima \cite{Maxima} или SymPy \cite{SymPy}). Это свойство наиболее ценно для возможности оперативного решения обратной задачи непосредственно во время эксперимента. 

\subsection{Каплевидное распределение} \label{subsec:model2}
Модели М1--М4 симметричны по высоте относительно точки с максимальной концентрацией. Нет оснований полагать, что реальное распределение концентрации возбуждённых атомов кислорода будет обладать подобной симметрией. Известно, что взаимодействие между волной накачки и ионосферной плазмой наиболее эффективно вблизи ее плазменных резонансов \cite{Trach1979}. С другой стороны распределение концентрации атомарного кислорода убывает с высотой $\propto \exp {(-h)}$, и, например, на высоте 200 км концентрация атомарного кислорода заметно выше, чем на высоте 250 км. Логично предположить, что модели М1--М4 могут хорошо описывать наблюдаемое свечение при размерах источника свечения меньших по сравнению с масштабом неоднородности среды. 

Рассмотрим модельное распределение М5, в общем случае несимметричное относительно максимума:
\begin{equation}\label{eq:drop1}
\begin{split}
n(x,y,z,\vec{p})=p_1 \exp{\bigg[-\frac{x^2}{p_2^2}-\frac{y^2}{p_3^2}\bigg]} \times \\
\times C(p_5,p_6) h(z) h(p_4-z) \times \\
\times \bigg[\frac{z}{p_4}\bigg]^{p_5-1}
\bigg[1-\frac{z}{p_4}\bigg]^{p_5-1},
\end{split}
\end{equation}
где параметры $p_{1-3}$ --- те же, что у модели М3, $p_4$ --- вертикальный размер области, а $p_5$ и $p_6$ характеризуют положение максимума концентрации на оси $z$. $h(z)$ --- функция Хевисайда. Из (\ref{eq:drop1}) видно, что $n(x,y,z,\vec{p})\neq 0$ при $z\in [0,p_4]$, а начало координат системы ENU$_\text{M}$ совпадает с нижней границей распределения концентрации. Распределение вдоль горизонтальной плоскости аналогично модели М3. Калибровочный коэффициент $C(p_5,p_6)$ обеспечивает максимальную концентрацию $p_1$:
\begin{equation}\label{eq:drop2}
\begin{split}
C(p_5,p_6)=\frac{(p_5+p_6)^2 -4(p_5+p_6)+4 }{p_5p_6-p_5-p_6+1} \times \\
\times \bigg[ \frac{p_5-1}{p_5+p_6-2}  \bigg] ^{p_5}
\bigg[ \frac{p_6-1}{p_5+p_6-2}  \bigg] ^{p_6}.
\end{split}
\end{equation}

\begin{figure}[h]
	\center{\includegraphics[width=1\linewidth]{fig6}}
	\caption{Зависимость концентрации возбуждённых атомов кислорода от высоты $z$ для модели М5 при $x=0, y=0$ и при различных значениях параметров $p_5$ и $p_6$.}
	\label{fig:fig6}
\end{figure}

При подстановке распределения (\ref{eq:drop1}) модели М5 в (\ref{eq:mod1}) аналитического выражения, как в случае моделей М1-М4, не получается. Но поскольку распределение в целом остаётся плавным, численное вычисление интеграла (например, с помощью метода Симпсона) не составляет труда.  

\section{Результаты решения обратной задачи} \label{sec:inv_problem_results}
Для однозначного определения системы координат ENU$_\text{M}$ (вне зависимости от выбранной модели) относительно системы ECEF необходимо добавить ещё три параметра $\varphi_\text{M}$, $\lambda_\text{M}$ и $h_\text{M}$. Объединим все параметры в один вектор $\vec{P}=\{\varphi_\text{M}, \lambda_\text{M}, h_\text{M}, \vec{p}\}$. Для любого вектора $\vec{P}$ путём моделирования (\ref{eq:mod2}) могут быть получены модельные изображения (матрицы поверхностных яркостей) $B_\text{А}^M$ и $B_\text{Б}^M$ соответствующие реальным изображениям $B_\text{А}$ и $B_\text{Б}$. Определим функцию ошибки следующим образом:
\begin{equation}\label{eq:task1}
\begin{split}
E(\vec{P})=\sum_{i,j}\big(B_{\text{А}}(i,j)-B_{\text{А}}^M(i,j)\big)^2+\sum_{i,j}\big(B_{\text{Б}}(i,j)-B_{\text{Б}}^M(i,j)\big)^2
\end{split}.
\end{equation}
Очевидно, что $E(\vec{P})=0$ при совпадении модельных изображений с реальными. Таким образом обратная задача определения параметров области свечения сводится к минимизации функции $E(\vec{P})$. Задача минимизации решалась методом Нелдера-Мида с помощью библиотеки SciPy \cite{SciPy}

Все зарегистрированные пары изображений анализировались с помощью моделей М1-М5. Сравнивая между собой значения функции ошибки, полученные для разных моделей, можно определить какая модель лучше согласуется с наблюдаемыми данными. Отметим, что из-за непостоянства прозрачности атмосферы, а также присутствия мелкой облачности, часть изображений пятен не была обработана. Решение обратной задачи было получено: 24.08.14 г. --- для 34 пар изображений, 26.08.14 г. --- для 11 пар и 29.08.16 г. --- для 2 пар. Поэтому, ввиду малого количества обработанных изображений, мы не приводим далее результатов за 29.08.16 г. Пример пары изображений пятен свечения с наложенными на них линиями уровня модельного распределения яркости (модель М5) приведен на рисунке \ref{fig:fig7}.
\begin{figure}[h]
	\center{\includegraphics[width=1\linewidth]{fig7}}
	\caption{Изображения пятен свечения зарегистрированных с помощью камер KEO (слева) и S1C (справа) 24.08.2014 в 19:29:30 UTC. Цветовая шкала поверхностной яркости задана в Рэлеях. Чёрными линиями уровня на панелях показаны модельные распределения. Использовалась модель М5 (каплевидное распределение) с параметрами: $\varphi_\text{M}=56,08^\circ$ с.ш., $\lambda_\text{M}=45,86^\circ$ в.д., $h_\text{M}=246$ км, $p_1=605$ см$^{-3}$, $p_2=23,3$ км, $p_3=28,5$ км, $p_4=82,4$ км, $p_5=3,84$ и $p_6=6$.}
	\label{fig:fig7}
\end{figure}
На рисунках \ref{fig:fig8_1}-\ref{fig:fig8_2} приведена зависимость от времени высоты центра модельного распределения концентрации ($h_\text{M}$) и высотного интервала, на котором концентрация спадает в $e$ раз. На этих же графиках представлены зависимости от времени высоты отражения $h_\text{r}$ и высоты верхнегибридного резонанса волны накачки $h_\text{UH}$, полученные с помощью ионозонда вертикального зондирования <<Циклон>> (КФУ). Для каждой пары реальных изображений на рисунках представлена только одна модель, дающая наименьшее значение $E(\vec{P})$. Среднее значение высоты области свечения в эксперименте 24.08.14 г. составило $249\pm12$ км, а 26.08.14 г. --- $256\pm5$ км. Какой-либо корреляции между $h_\text{r}$ ($h_\text{ВГ}$) и $h_\text{M}$ не наблюдается. Во время эксперимента 24.08.14 г. $h_\text{r}$ плавно увеличивается с $\sim253$ до $\sim270$ км, а $h_\text{ВГ}$ отстоит от $h_r$ на $\sim10$ км вниз.
Во время эксперимента 26.08.14 г. высота отражения увеличивается приблизительно на 40 км (с $\sim285$ км до $\sim310$ км), в то время как $h_\text{M}$ в среднем практически не изменяется. Средняя ширина диапазона высот (по уровню $1/e$ от максимальной концентрации), которое занимает светящаяся область, составляет: 24.08.14 г. --- $28\pm18$ км, 26.08.14 г. --- $39\pm 11$ км. 
\begin{figure}[h]
	\center{\includegraphics[width=1\linewidth]{fig8_1}}
	\caption{Динамика высоты центра пятна в эксперименте, проведённом 24.08.14 г. Значения высоты $h_\text{M}$ отмечены маркерами. Тип маркера соответствует определённой модели (М3, М4 или М5). Рядом с каждым маркером сплошной линией показан интервал высот, на котором концентрация возбуждённых атомов кислорода выше уровня $1/e$ от максимальной. Синей сплошной (пунктирной) линией показана динамика высоты отражения волны накачки $h_\text{r}$ (верхнегибридного резонанса волны накачки $h_\text{ВГ}$). Чёрной линией в нижней части рисунка показаны интервалы включения волны накачки.}
	\label{fig:fig8_1}
\end{figure}	

\begin{figure}[h]
	\center{\includegraphics[width=1\linewidth]{fig8_2}}
	\caption{То же, что на рисунке \ref{fig:fig8_1} для эксперимента, проведенного 26.08.2014 г.}
	\label{fig:fig8_2}
\end{figure}	

На рисунке \ref{fig:fig9_1} показано изменение максимальной концентрации $p_1$ в ходе эксперимента, проведённого 24.08.14 г. Среднее значение концентрации составило: 24.08.14 г. --- $1100\pm900$ см$^{-3}$, 26.08.14 г. --- $1300\pm1100$ см$^{-3}$. Также как и в случае $h_\text{M}$, в течение экспериментов не наблюдалось какого-либо статистически значимого изменения максимальной концентрации.

\begin{figure}[h]
	\center{\includegraphics[width=1\linewidth]{fig9_1}}
	\caption{Динамика максимальной концентрации возбужденных атомов кислорода в пятне искусственного свечения (параметр $p_1$ модели) в эксперименте, проведённом 24.08.14 г. Значения концентрации отмечены маркерами. Тип маркера соответствует определённой модели (М3, М4 или М5). Линиями около маркеров показана абсолютная погрешность. Чёрной линией в нижней части рисунка показаны интервалы включения волны накачки.}
	\label{fig:fig9_1}
\end{figure}

Среднее отношение вертикального и горизонтального размеров области свечения (коэффициент вытянутости) составило: 24.08.14 г. --- $1,00\pm0,86$, 26.08.14 г. --- $1,02\pm0,62$. Можно сказать, что в среднем область свечения не была вертикально вытянутой, хотя в нескольких случаях коэффициент вытянутости был >2. Направление оси сфероидов модели М4, характеризуемое параметрами $p_4$ и $p_5$, значительно менялось от изображения к изображению. Какой-либо связи между ним и направлением геомагнитного поля на высотах свечения не обнаружено. 

\section{Обсуждение результатов} \label{sec:discuss}

В результате решения обратной задачи из $\sim 130$ пар изображений было отобрано 47, для которых были определены параметры светящейся области. Погрешность разработанной методики определяется ошибками астрометрической и спектрофотометрической калибровок. 
На основе величин ошибок соответствующих калибровок, приведённых в разделах \ref{sec:astro_cal} и \ref{sec:spectro_cal}, можно показать, что пространственные параметры моделей определяются с погрешностью не превышающей $\sim700$ м, а концентрация возбуждённых атомов --- не выше $250$ см$^{-3}$. 
%При этом методика очень чувствительна к входным данным. 
Зарегистрированная поверхностная яркость искусственного свечения в экспериментах, проведённых на стенде <<Сура>> 24.08.14 г., 26.08.14 г. и 29.08.16 г. не превышала 25 Рэлей.
%, чего, по-видимому, недостаточно для уверенного определения параметров области свечения из-за шумов, связанных с неравномерным распределением коэффициента прозрачности по кадру. С этим может быть связано большое число <<отбракованных>> пар изображений, а также большая дисперсия отдельных модельных параметров.
%, чем в экспериментах, упоминаемых в, \cite{Grach2012,Shindin2014,Klimenko2017}.
%В рассмотренных экспериментах по-видимому часто интенсивность свечения приближалась к уровню шумов, связанных с неравномерным распределением коэффициента прозрачности по кадру, тем самым затрудняя решение обратной задачи. 
%Свой вклад внесли также значительные различия в технических характеристиках используемых камер. Кроме того, расположение приемных пунктов было не оптимальным. При увеличении количества разнесенных приемных пунктов неопределенность в определении параметров модели должна уменьшиться.

%Длина релаксации энергичных электронов в зависимости от высоты
%(Гуревич, Димант, Милих, Васьков 1985) для 300 км - 100 км, на 250 км - 20 км

В рассмотренных экспериментах высота центра светящейся области не зависела от резонансных высот волны накачки.
%, что соответствует \cite{Gustavsson2001}.
Согласно \cite{Bernhardt1989,Kosch2002} оптическое свечение генерируется на высотах $220\div270$ км, где достаточно высока концентрация атомов кислорода, тогда как ускорение электронов до энергий $\varepsilon > 2$ эВ происходит в области плазменных резонансов волны накачки. При подъёме области ускорения выше 270 км, электронам для возбуждения достаточного числа атомов кислорода и перевода их в состояние O($^1$D) необходимо попасть в более низкие и, следовательно, плотные слои атмосферы.
В \cite{Vaskov1983,Gurevich1985} проведён численный расчёт длины релаксации $L$ электронов с энергией 4,5 эВ в зависимости от высоты источника электронов. В частности показано, что $L=70,40$ и 20 км на высотах 300, 275 и 250 км соответственно. 
Т.о. генерация свечения должна наиболее эффективно происходить на высотах 240-260 км, что и имеет место в эксперименте. Кроме того, высыпанием ускоренных электронов с высот ускорения (>270 км) в область генерации свечения (240-260 км) интерпретировалось смещение пятна свечения на север в экспериментах \cite{Grach2012}. Эти результаты качественно согласуются с наблюдаемыми в экспериментах высотами центра области свечения.
%Это косвенно подтверждает наблюдаемое постоянство высоты центра области свечения. 

Форма распределения концентрации возбуждённых атомов кислорода наиболее хорошо описывается моделями М4 (изоповерхности --- наклонные сфероиды) или М5 (каплевидное распределение). Большинство наблюдаемых пятен лучше описываются моделью М4 (в $\sim40\%$ случаев), однако какой-либо выделенной ориентации светящихся областей не выявлено. Это может быть связано, во-первых, с малой наблюдаемой интенсивностью свечения и, во-вторых, с недостаточным угловым разрешением камеры KEO. Ожидаемая ориентация области свечения вдоль геомагнитного поля (отклонение от зенита --- $18,1^\circ$) не даёт при проектировании области на поле зрения камеры заметных отличий от случая вертикально направленной области

Модель М5 лучше согласовывалась с наблюдаемыми пятнами в $\sim40\%$ случаев в эксперименте 24.08.14 г. При этом в 6 случаях из 10-ти высотное распределение было близко к симметричному относительно высоты с максимальной концентрацией, а в 4-х случаях --- максимум концентрации был значительно сдвинут вниз (см. синюю кривую на рисунке \ref{fig:fig6}). Можно предположить, что если к модели М5 добавить два угловых параметра, аналогичных параметрам $p_4$ и $p_5$ модели М4, характеризующих ориентацию модели, то получившаяся модель будет лучше подходить для описания наблюдаемых эффектов, чем все рассмотренные в данной работе модели, хотя и ценой многократного увеличения времени решения обратной задачи.

Полученная в результате решения обратной задачи максимальная концентрация возбуждённых атомов кислорода ($\sim1000$ см$^{-3}$) согласуется с оценками, приведёнными в \cite{Klimenko2017}.


~\

Рисунки для публикации подготовлены с помощью библиотеки Mat\-plotlib \cite{Hunter2007}. С исходными кодами разработанных алгоритмов вместе с данными для построения рисунков можно найти в \cite{Shindin2017}.
Работа выполнена при поддержке Российского научного фонда: основная часть --- соглашение № 14-12-00706П; разработка каплевидной модели (раздел \ref{subsec:model2}) и обработка данных с её помощью --- соглашение № 17-72-10181). Экспериментальные данные получены с использованием оборудования ЦКП <<Ангара>> (ИСЗФ СО РАН). 

\renewcommand{\thesection}{\Asbuk{section}}

\appendix
\titleformat{\section}[display]
{\normalfont\Large\bfseries}{Приложение \thesection}{0pt}{\Large}

\section{Астрометрическая калибровка изображений звёздного неба с учетом типа проекции небесной сферы} \label{app:astrometric}

Для определения математической связи между двумя системами координат ($xy$ связанной с изображением и одной из небесных) вводится дополнительная плоская система координат $XY$ (т.н. стандартные или идеальные координаты), которая получается из $xy$ путём линейного преобразования. Физический смысл начала координат $XY$ - точка пересечения оптической оси системы с плоскостью сенсора камеры. Связь между системами $xy$ и $XY$ можно записать с помощью двух следующих систем:

\begin{equation}\label{eq:eq1}
\begin{cases}
X = a_1 + a_2 x +a_3 y\\
Y = b_1 + b_2 x +b_3 y,\\
\end{cases}
\end{equation}

\begin{equation}\label{eq:eq2}
\begin{cases}
x = c_1 + c_2 X +c_3 Y\\
y = d_1 + d_2 X +d_3 Y.\\
\end{cases}
\end{equation}

Вид математической связи между системой $XY$ и небесной системой координат:
\begin{equation}\label{eq:eq3}
\begin{cases}
A = A_0 - \arctg\frac{\cos h_i\sin A_i}{\sin h_i \cos h_0 -\cos h_i \sin h_0 \cos A_i}\\
h = \arcsin (\sin h_i \sin h_0 + \cos h_i \cos h_0 \cos A_i),
\end{cases}
\end{equation}
\begin{equation}\label{eq:eq5}
A_i = \arctg\frac{X}{Y}, R = \sqrt{X^2 + Y^2}.
\end{equation} 
Здесь выражение для $h_i$ зависит от типа проекции, используемой в камере \cite{Calabretta2002}. Для гномонической проекции (камеры S1C и SBIG): 
\begin{equation}\label{eq:eq7}
h_i = \arctg\frac{180}{\pi R},
\end{equation}
а для зенитной эквидистантной проекции (камера KEO): 
\begin{equation}\label{eq:eq8}
h_i = \frac{\pi}{180}(90-R).
\end{equation}
Для обратного перехода к системе $XY$ из горизонтальной системы координат: в случае гномонической проекции ---
\begin{equation}\label{eq:eq8_dop}
\begin{cases}
X=\frac{-180}{\pi}\frac{\cos{h}\sin{(A-A_0)}}{\cos{h_0}\cos{h}\cos{(A-A_0)}+\sin{h_0}\sin{h}}\\
Y=\frac{180}{\pi}
\frac{\sin{h_0}\cos{h}\cos{(A-A_0)}-\cos{h_0}\sin{h}}
{\cos{h_0}\cos{h}\cos{(A-A_0)}+\sin{h_0}\sin{h}},
\end{cases}
\end{equation}
а в случае зенитной эквидистантной проекции ---
\begin{equation}\label{eq:eq9_dop}
\begin{cases}
h_i=\arcsin{(\sin{h}\sin{h_0}+\cos{h}\cos{h_0}\cos{(A-A_0)})}\\
A_i=-\arctg{\frac
	{\cos{h}\sin{(A-A_0)}}
	{\sin{h}\cos{h_0}-\cos{h}\sin{h_0}\cos{(A-A_0)}}},
\end{cases}
\end{equation}
\begin{equation}\label{eq:eq10_dop}
\begin{cases}
X= (90-h_i\frac{180}{\pi})\sin{A_i}\\
Y= (90-h_i\frac{180}{\pi})\cos{A_i}.
\end{cases}
\end{equation}
Коэффициенты $a_{1-3}$, $b_{1-3}$, $c_{1-3}$, $d_{1-3}$ определяются из таблицы координат опорных звёзд кадра методом наименьших квадратов, а параметры $A_0, h_0$, характеризующие направление оптической оси камеры, определяются путём минимизации функции ошибки, возвращающей среднее расстояние в пикселях между истинными плоскими координатами опорных звёзд на снимке и вычисленными координатами, рассчитанными по стандартным координатам с помощью набора параметров $c_{1-3}$, $d_{1-3}$.
Описанная методика работает и в случае небесных экваториальных координат: достаточно заменить в формулах (\ref{eq:eq3}-\ref{eq:eq5},\ref{eq:eq7}-\ref{eq:eq8}) азимут $A$ и высоту $h$ на прямое восхождение $\alpha$ и склонение $\delta$.

Поскольку ориентация камер относительно земной поверхности не менялась в течение эксперимента, коэффициенты $a_{1-3}$, $b_{1-3}$, $c_{1-3}$, $d_{1-3}$ и параметры $A_0, h_0$ так же не должны меняться от кадра к кадру. И так как для камер S1C и SBIG процесс астрометрической калибровки автоматизирован, набор коэффициентов и параметров для каждого кадра усреднялся по достаточно большой выборке (до 700 кадров).

\section{Учет проекции небесной сферы при повороте изображения звёздного неба} \label{app:postprocessing}
Представим изображение в виде матрицы $B$ размером $m \times n$ и введём матрицы $X_{\text{M}}$ и $Y_{\text{M}}$ того же размера $m \times n$:
\begin{equation}\label{eq:eq16}
X_{\text{M}}=
\begin{bmatrix}
1 & 2 & 3 & \dots  & n \\
1 & 2 & 3 & \dots  & n \\
\vdots & \vdots & \vdots & \ddots & \vdots \\
1 & 2 & 3 & \dots  & n \\
\end{bmatrix}, 
Y_{\text{M}}=
\begin{bmatrix}
1 & 1 & 1 & \dots  & 1 \\
2 & 2 & 2 & \dots  & 2 \\
\vdots & \vdots & \vdots & \ddots & \vdots \\
m & m & m & \dots  & m \\
\end{bmatrix},
\end{equation}
характеризующие координаты пикселя в системе координат $xy$, связанной с плоскостью кадра. Тогда пиксель изображения c координатами  $(X_{\text{M}ij},Y_{\text{M}ij})$ соответствует элементу матрицы $B_{ij}$. В результате астрометрической калибровки (см. формулы (\ref{eq:eq1}),(\ref{eq:eq3}-\ref{eq:eq8})) можно перейти от плоских координат к горизонтальным небесным координатам и получить матрицы азимута $A_{\text{M}}$ и высоты над горизонтом $h_{\text{M}}$. Не составляет труда, зная время $t$ и место регистрации изображения,	перейти затем к экваториальным координатам и получить матрицы прямого восхождения $\alpha_{\text{M}}$ и склонения $\delta_{\text{M}}$. Таким образом элементу матрицы изображения $B_{ij}$ будут соответствовать угловые координаты $(A_{\text{M}ij},h_{\text{M}ij})$ и $(\alpha_{\text{M}ij},\delta_{\text{M}ij})$. Так как на рассматриваемых временных отрезках (несколько минут) экваториальные координаты звёзд можно считать постоянными, а горизонтальные наоборот меняются со временем, то всегда можно выполнить обратный переход от экваториальных координат к горизонтальным $(\tilde{A}_{\text{M}},\tilde{h}_{\text{M}})$, задав время $\tilde{t}\neq t$ отличное от времени регистрации изображения. Координаты звёзд в новой горизонтальной системе координат $(\tilde{A}_{\text{M}},\tilde{h}_{\text{M}})$ будут соответствовать моменту времени $\tilde{t}$. Следующий этап --- обратный переход к плоским координатам $(\tilde{X}_{\text{M}},\tilde{Y}_{\text{M}})$ по формулам (\ref{eq:eq2}),(\ref{eq:eq8_dop}-\ref{eq:eq10_dop}). Вид матриц $\tilde{X}_{\text{M}}$ и $\tilde{Y}_{\text{M}}$ после вышеописанной цепочки преобразований уже будет не таким простым как у $X_{\text{M}}$ и $Y_{\text{M}}$ в (\ref{eq:eq16}). Совокупность трёх матриц $X_{\text{M}}$, $Y_{\text{M}}$ и $B$ можно рассматривать как поверхность, заданную на неравномерной сетке значений $X_{\text{M}ij}$, $Y_{\text{M}ij}$. Для того, чтобы была возможность работать с этой поверхностью как с изображением, необходимо привести её к исходной сетке \ref{eq:eq16}. Это можно сделать путем триангуляции и интерполяции. Полученную матрицу значений (новое изображение) обозначим $\tilde{B}$. Вышеописанную процедура применялась при генерации фонового изображения для двух опорных кадров, ближайших к текущему кадру. 

\section{Преобразования систем координат} \label{app:modelling1}
\begin{equation}\label{eq:mod6}
M_1=\begin{pmatrix} -\sin{\lambda_\text{M}} & \cos{\lambda_\text{M}} & 0 \\ -\cos{\lambda_\text{M}}\sin{\varphi_\text{M}} & -\sin{\lambda_\text{M}}\sin{\varphi_\text{M}} & \cos{\varphi_\text{M}}\\ \cos{\lambda_\text{M}}\cos{\varphi_\text{M}} & \sin{\lambda_\text{M}}\cos{\varphi_\text{M}} & \sin{\varphi_\text{M}} \end{pmatrix},
\end{equation}

\begin{equation}\label{eq:mod7}
M_2^\text{А}=\begin{pmatrix} -\sin{\lambda_\text{А}} & 
-\cos{\lambda_\text{А}}\sin{\varphi_\text{А}} &
\cos{\lambda_\text{А}}\cos{\varphi_\text{А}} \\
\cos{\lambda_\text{А}} &
-\sin{\lambda_\text{А}}\sin{\varphi_\text{А}} &
\sin{\lambda_\text{А}}\cos{\varphi_\text{А}} \\
0 & \cos{\varphi_\text{А}} & \sin{\varphi_\text{А}} \end{pmatrix},
\end{equation}
а $M_2^\text{Б}$ получается из (\ref{eq:mod7}) заменой $\text{А} \rightarrow \text{Б}$.

%Таким образом из исходной функции (\ref{eq:mod3}) мы получим две: $n_\text{А}=n_\text{А}(x,y,z)$, $n_\text{Б}=n_\text{Б}(x,y,z)$, заданные в системах координат ENU$_\text{А}$ и ENU$_\text{Б}$ соответственно. 
Переход к угловым координатам ($A,h,r$) осуществляется с помощью системы:
\begin{equation}\label{eq:mod8}
\begin{cases}
x = r\cos{h}\sin{A} \\
y = r\cos{h}\cos{A} \\
z = r\sin{h}
\end{cases}.
\end{equation}

\bibliography{citations}

\end{document}